\section{Class 1 - Set Theory, Probability, and Indicator Functions}

\subsection{Set Theory}

\subsubsection{Review of basic definitions}
We begin with definitions that should be familiar. 

\begin{definition}[Set]: A \textbf{set} is a collection of elements.
\end{definition}

\begin{definition}[Subset and superset]: A set $A$ is a \textbf{subset} of $B$ if every element of $A$ is also an element of $B$, denoted 
  \[
      A \subset B
  \]
  Equivalently, $B$ is a \textbf{superset} of $A.$
\end{definition}

\begin{definition}[Null set and empty set] The set with no elements is called the \textbf{null set} or the \textbf{empty set}, denoted $\emptyset$.
\end{definition}

\begin{remark}
  The null set is a subset of any set. I.e. for any $A$,
  \[
      \emptyset \subset A
  \]
\end{remark}

\begin{definition}
  (Universal set): The \textbf{universal set} is the set of all things that we could possibly consider in the context we are studying. 
\end{definition}

\begin{remark}
  In probability, the universal set is typically the sample space denoted $\Omega$
\end{remark}

\subsubsection{Review of set operations}
Except for \textbf{symmetric difference}, most of these set operations should be familiar.

\begin{definition}[Union]: The \textbf{union} of two sets, $A$ and $B$, is a set containing all the elements that are in $A$ or in $B$ (possibly both). \\

  The union of two sets, $A$ and $B$ is denoted 
  \[
  A \cup B
  \]
  The union of three or more sets, say $A_1, A_2, \hdots A_n$ is denoted 
  \[
      A_1 \cup A_2 \cup A_3 \hdots \cup A_n = \bigcup_{i = 1}^n A_i
  \]
\end{definition}

\begin{definition}
  [Intersection]: The \textbf{intersection} of two sets, $A$ and $B$, is a set containing all the elements that are both in $A$ and $B$. \\

  The intersection of two sets, $A$ and $B$ is denoted 
  \[
  A \cap B
  \]
  The intersection of three or more sets, say $A_1, A_2, \hdots A_n$ is denoted 
  \[
      A_1 \cap A_2 \cap A_3 \hdots \cap A_n = \bigcap_{i = 1}^n A_i
  \]
\end{definition}

\begin{definition}
  [Complement]: The \textbf{complement} of a set $A$ denoted by $A^C$ is the set of all elements that are in the universal set $S$ but are not in $A$.
\end{definition}

\begin{definition}
  [Difference, subtraction of sets]: The \textbf{difference} of two sets is defined where $A - B$ consists of the elements that are in $A$ but not in $B$. This is denoted
  \[
      A - B = A \setminus B = \{ x \in \Omega: x \in A, x \notin B \} 
  \]
\end{definition}

\begin{remark}
  From the above definition, it should be clear that 
  \[
      A - B = A \cap B^C
  \]
\end{remark}

\begin{definition}
  [Mutually exclusive, disjoint]: Two sets, $A$ and $B$, are \textbf{mutually exclusive} or \textbf{disjoint} if they do not have any shared elements, i.e. 
  \[
      A \cap B = \emptyset
  \]
\end{definition}

For three or more sets, the sets having a trivial intersection does not mean they are disjoint. Instead, we require the stronger condition below.

\begin{definition}
  [Pairwise disjoint]: Several sets are \textbf{pairwise disjoint} if no two sets share a common element.
\end{definition}

% \begin{definition}
%   (Partition): We say that a collection of nonempty sets $A_1, A_2, \hdots$ form a \textbf{partition} of $A$ if they are disjoint and their union is $A$.
% \end{definition}


\subsubsection{Review of common set properties}

\begin{theorem}
  (De Morgan's law): For sets $A_1, A_2, \hdots A_n$, we have 
  \begin{itemize}
      \item $ \left( A_1 \cup A_2 \cup \hdots A_n \right)^C = A_1^C \cap A_2^C \cap \hdots A_n^C $
      \item $ \left( A_1 \cap A_2 \cap \hdots A_n \right)^C = A_1^C \cup A_2^C \cup \hdots A_n^C $
  \end{itemize} 
\end{theorem}

\begin{theorem}
  (Distributive law): For any sets $A, B, C$, we have 
  \begin{itemize}
    \item $A \cap (B \cup C) = (A \cap B) \cup (A \cap C)$
    \item $A \cup (B \cap C) = (A \cup B) \cap (A \cup C)$
  \end{itemize}
\end{theorem}

\subsubsection{The symmetric difference}
Pay especially close attention to the symmetric difference, which may not have been emphasized upon in previous courses.

\begin{definition}
  [Symmetric difference]: The \textbf{symmetric difference} of two sets, $A$ and $B$, is defined as the set of elements that are only in $A$ or in $B$, but not both. This is denoted
  \[
  A \symd B
  \]

  The symmetric difference of two sets is their union, minus their intersection 
  \[
      A \symd B = (A \cup B) \setminus (A \cap B)
  \]
\end{definition}

\begin{result}
  Properties of the symmetric difference
  \begin{enumerate}
      \item Commutativity
      \[
      A \symd B = B \symd A
      \]
      \item Associativity
      \[
      (A \symd B) \symd C = A \symd (B \symd C)
      \]
      \item Distributivity of the intersection 
      \[
          A \cap \left( B \symd C \right)  = \left( A \cap B \right)  \symd (A \cap C)
      \]
      \item The symmetric difference is trivial if and only if the two sets are equal 
      \[
          A \symd B = \emptyset \iff A = B
      \]
      \item Taking complement with respect to the same universal set, 
      \[
      A \symd B = A^C \symd B^C
      \]
      \item The symmetric difference is a subset of the union
      \[
      A \symd B \subseteq A \cup B
      \]
      \item The symmetric difference is equal to the union if and only if the sets are disjoint 
      \[
      A \symd B = A \cup B \iff A \cap B = \emptyset
      \]
      \item The symmetric difference and the intersection partition the union, since
      \begin{align*}
          & (A \symd B) \cap (A \cap B) = \emptyset  \\
          & (A \symd B) \cup (A \cap B) = A \cup B
      \end{align*}
      \textit{The concept of partition will be introduced in Class 2}
      \item We can define the union using 
      \[
          A \cup B = (A \symd B) \symd (A \cap B)
      \]
  \end{enumerate} 
\end{result}

The following result is especially important. Hence we discuss it separately. 

\begin{lemma}
  \label{lem:symm-diff-union}
  Given arbitrary sets $A_1, A_2, \hdots A_n$ and $B_1, B_2, \hdots B_n$, 
  \begin{align*}
    \left( \bigcup_{i = 1}^n A_i \right)  \symd \left( \bigcup_{i = 1}^n B_i \right)  \subseteq \bigcup_{i = 1}^n \left( A_i \symd B_i \right) 
  \end{align*}
\end{lemma}

The proof of this lemma is revisited later in the lecture after the introduction of indicator functions. We will see that the proof is simple once we introduce indicators.

\subsection{Probabilities}

\subsubsection{Review}

\begin{definition}
  [Random experiment, outcome, sample space]: 
  \begin{itemize}
      \item A \textbf{random experiment} is a process by which we observe something uncertain.
      \item The result of a random experiment is an \textbf{outcome}. 
      \item The set of possible outcomes is the sample space.
  \end{itemize} 
\end{definition}

\begin{definition}
  [Event]: An \textbf{event} $E$ is a subset of the sample space, i.e. a collection of outcomes.
\end{definition}

\begin{remark}
  If $A$ and $B$ are events, then $A \cup B$ and $A \cap B$ are also events. \\

  $A \cup B$ occurs if $A$ \textbf{or} $B$ occurs. \\

  $A \cap B$ occurs if $A$ \textbf{and} $B$ occurs. \\
\end{remark}

\begin{definition}
  [Probability]: The \textbf{probability} measure of event $A$ is denoted $P(A)$.
\end{definition}


\subsubsection{Axioms of probability}
\begin{definition}
  [Axioms of probability]: The axioms of probability state that 
  \begin{enumerate}
      \item For any event $A$, $P(A) \geq 0$
      \item Probability of the sample space $\Omega$ is $P(\Omega) = 1$
      \item If $A_1, A_2, A_3 \hdots$ are disjoint, then 
      \[
      P(A_1 \cup A_2 \cup A_3 \hdots) = P(A_1) + P(A_2) + P(A_3) + \hdots
      \]
  \end{enumerate} 
\end{definition}

\subsubsection{Inclusion-Exclusion Principle}

\begin{result}
  [Inclusion-exclusion]: By the \textbf{inclusion-exclusion principle}, we have 
  \[
  P(A \cup B) = P(A) + P(B) - P(A \cap B)
  \]
  In general for $n$ events $A_1, \hdots A_n$, 
  \begin{align*}
  P \left( \cup_{i =1}^n A_i \right)  &= \sum\limits_{i = 1}^{n}  P(A_i) \\
& - \sum\limits_{i < j}^{}  P(A_i \cap A_j) \\
& + \sum\limits_{i < j < k}^{}  P(A_i \cap A_j \cap A_k) \\
& \vdots \\
& + (-1)^{n-1} P\left(\bigcap_{i = 1}^n A_i\right) \\
  \end{align*}
\end{result}



\subsection{Indicator Functions}

\subsubsection{Definition}

\begin{definition}
  [Indicator function]: Given an arbitrary set $X$, and a subset $A \subseteq X$, the \textbf{indidcator function} of $A$ is 
  \[
  \bm{1}_A(x) = \begin{cases}
    1 \text{ if } x \in A \\
    0 \text{ if } x \notin A
  \end{cases}
  \]
\end{definition}

\subsubsection{Properties of the indicator function}

\begin{result}
  Properties of the indicator function 
  \begin{enumerate}
    \item Indicator of the intersection is the product of indicators
    \[
    \bm{1}_{A \cap B} (x) = \min \{ \bm{1}_A(x), \bm{1}_B(x) \} = \bm{1}_A(x) \cdot \bm{1}_B(x)
    \]
    \item Ihe indicator of the union is sum of indicators minus their product
    \begin{align*}
    \bm{1}_{A \cup B} (x) &= \max \{ \bm{1}_A(x), \bm{1}_B(x) \} \\
    & = \bm{1}_A(x) + \bm{1}_B(x) - \bm{1}_A(x) \cdot \bm{1}_B(x) \\
    & = \bm{1}_A(x) + \bm{1}_B(x) - \bm{1}_{A \cap B}(x) \\
    \end{align*}
    \item Indicator of the complement 
    \[
    \bm{1}_{A^C} = 1 - \bm{1}_A
    \]
    \item If $A, B$ disjoint 
    \begin{align*}
    \bm{1}_{A \cup B} &= \bm{1}_{A} + \bm{1}_B \\
    \bm{1}_{A \cap B} &= 0
    \end{align*}
    \item Indicators of subsets
    \[
    A \subseteq B \iff \bm{1}_A \leq \bm{1}_B
    \]
    \item Indicators of difference of subsets
    \begin{align*}
      \bm{1}_{A - B}
      &= \bm{1}_{A \cap B^C} \quad \text{by definition of set subtraction} \\
      &= \bm{1}_A \cdot \bm{1}_{B^C} \quad \text{by indicator of intersections} \\
      &= \bm{1}_A\bigl(1 - \bm{1}_B\bigr) \quad \text{by indicator of complement} \\
      &= \bm{1}_A - \bm{1}_{A \cap B} \quad \text{by indicator of intersections}
    \end{align*}
    \item Indicators of symmetric difference
    \begin{align*}
      \bm{1}_{A \symd B}
      &= \bm{1}_{(A \cup B) \setminus (A \cap B)}
        \quad \text{by definition of symmetric difference} \\[6pt]
      &= \bm{1}_{A \cup B} \cdot \bm{1}_{(A \cap B)^C}
        \quad \text{by definition of set subtraction} \\[6pt]
      &= \bm{1}_{A \cup B}\bigl(1 - \bm{1}_{A \cap B}\bigr)
        \quad \text{by indicator of complement} \\[6pt]
      &= \bigl(\bm{1}_A + \bm{1}_B - \bm{1}_{A \cap B}\bigr)
        \bigl(1 - \bm{1}_{A \cap B}\bigr)
        \quad \text{by indicator of union} \\[6pt]
      &= \bm{1}_A + \bm{1}_B - 2\,\bm{1}_{A \cap B}
        \quad \text{since } \bm{1}_{A \cap B}^2 = \bm{1}_{A \cap B} \\[6pt]
      &= \bigl|\bm{1}_A - \bm{1}_B\bigr|
    \end{align*}
  \end{enumerate}
\end{result}

\subsubsection{Demonstrating the usefulness of indicators}
Recall Lemma~\ref{lem:symm-diff-union}. Given arbitrary sets $A_1,\dots,A_n$ and $B_1,\dots,B_n$,
\[
\left( \bigcup_{i=1}^n A_i \right)
\symd
\left( \bigcup_{i=1}^n B_i \right)
\subseteq
\bigcup_{i=1}^n (A_i \symd B_i).
\]

We can prove this lemma by reducing the set inclusion to an inequality involving indicator functions.

\begin{proof}
  By property 1.23.5, it suffices to show 
\begin{equation*}
\bm{1}_{\left(\cup_{i=1}^n A_i\right)\symd\left(\cup_{i=1}^n B_i\right)}
\le
\bm{1}_{\cup_{i=1}^n (A_i \symd B_i)}
\end{equation*}

By property 1.23.7 (indicator of symmetric differences), the LHS can be written as
\[
\left|
\bm{1}_{\cup_{i=1}^n A_i}(x)
-
\bm{1}_{\cup_{i=1}^n B_i}(x)
\right|
\]

By property 1.23.2 (indicator of unions)
\[
\bm{1}_{\cup_{i=1}^n A_i}(x) = \max_{1 \le i \le n} \bm{1}_{A_i}(x),
\qquad
\bm{1}_{\cup_{i=1}^n B_i}(x) = \max_{1 \le i \le n} \bm{1}_{B_i}(x),
\]
Hence we get 
\begin{equation*}
\left|
\max_{1 \le i \le n} \bm{1}_{A_i}(x)
-
\max_{1 \le i \le n} \bm{1}_{B_i}(x)
\right|
\le
\bm{1}_{\cup_{i=1}^n (A_i \symd B_i)}(x)
\end{equation*}

We now prove this inequality by enumerating the possible values of the
right-hand side.

\medskip

\noindent
\textbf{Case 1:} RHS = 0
\begin{align*}
& \bm{1}_{\cup_{i=1}^n (A_i \symd B_i)}(x) = 0 \\
\implies  & x \notin A_i \symd B_i \text{ for all } i \\
\implies & \bm{1}_{A_i}(x) = \bm{1}_{B_i}(x) \text{ for all } i \\
\implies &  \max_{i} \bm{1}_{A_i}(x) = \max_{i} \bm{1}_{B_i}(x) \\
\implies & 
\left|
\max_{1 \le i \le n} \bm{1}_{A_i}(x)
-
\max_{1 \le i \le n} \bm{1}_{B_i}(x)
\right| = 0
\end{align*}
Hence the inequality holds.

\medskip

\noindent
\textbf{Case 2:} RHS = 1 

Since the LHS is the absolute value of the difference of indicators, it takes values $0$ or $1$, and this is a simple upper bound.

\medskip

In both cases, the inequality holds. Since this inequality is
equivalent to the desired set inclusion, the lemma follows.
\end{proof}
\newpage